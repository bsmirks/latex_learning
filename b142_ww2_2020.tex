\documentclass{article}
\usepackage[margin=1in]{geometry}

\title{The Role of Field Artillery in the European Theatre of WWII}
\author{Blake Smreker}
\date{\today}

\begin{document}
\maketitle

Field Artillery (FA) in World War II was directly responsible for turning the tide of the battle and the war as a whole.
While the term was coined in WWI, the European Theatre solidified the notoriety FA earned to give them the nickname
`King of Battle' due to the fact that it was directly responsible for producing the highest amount of casualties in a
weapon system that had ever been seen. In Europe's eastern front alone, artillery caused 70~percent of German
casualties. If one were to compare a quote to the extensive power that Field Artillery possesses, the closest one would be
from the Bhagavad Gita: ``Now I am become death, Destroyer of Worlds.''

The word ``artillery'' actually refers to several moving parts. The word is a result of having good quality stock of
ammunition, guns, observers, fire direction, and the ability to lay accurate, rapid fires at the enemy. All of these pieces
must move harmoniously between each other in order to win. In Europe, the U.S.\ Army was known to the Axis forces for having
the most efficient artillery to date, acknowledging that they were outplayed each time not by building bigger guns, but by
focusing on the art of perfecting the current weaponry. If one were to visit the Field Artillery museum at Fort Sill, the
home of FA in the U.S military, the evolution of these systems is evident primarily by observing the transition between
primitive weaponry, like the cannons used by Napoleon Bonaparte's hordes, to the guns used in WWII, which highlights the
usefulness of have a machine that focuses on efficient recoil systems which ultimately allowed for more efficient rapid fires.
Another obvious example of the American's implementation of superior firepower in this area would be the transition mid-war
from the French 75mm and 105mm purchased after the first Great War to the famous M2/M3 105mm howitzer and the M1 75mm pack howitzer
that effectively kickstarted a renaissance of artillery implementation in Airborne, Mountain, and Jungle divisions.

The M1 75mm pack howitzer, later redesignated to the M116 75mm howitzer, could be towed by any small vehicle and allowed for
highly mobile units such as the Airborne divisions to leave combat as fast as they could enter. Due to it's modularity, the M1
could very easily be broken down and attached to parachutes and dropped behind enemy lines with infantry divisions, leading to
devastating mortality rates and countless successful battles and an almost iminent deterioration of the enemy's morale. 

\end{document}
