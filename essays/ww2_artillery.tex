\documentclass[12pt]{article}
\usepackage[margin=1in]{geometry}
\renewcommand{\linespread}{1.0}

\setlength{\parindent}{1cm}

\title{The Role of Field Artillery in the European Theatre of WWII}
\author{Blake Smreker}
\date{\today}

\begin{document}
\maketitle

Field Artillery (FA) in World War II was directly responsible for turning the tide of the battle and the war as a whole.
While the term was coined in WWI, the European Theatre solidified the notoriety FA earned to give them the nickname
`King of Battle' due to the fact that it was directly responsible for producing the highest amount of casualties in a
weapon system that had ever been seen. In Europe's eastern front alone, artillery caused 70~percent of German
casualties. If one were to compare a quote to the extensive power that Field Artillery possesses, the closest one would be
from the Bhagavad Gita: ``Now I am become death, Destroyer of Worlds.''

\vspace{1.0\baselineskip}

The word ``artillery'' actually refers to several moving parts. The word is a result of having good quality stock of
ammunition, guns, observers, fire direction, and the ability to lay accurate, rapid fires at the enemy. All of these pieces
must move harmoniously between each other in order to win. In Europe, the U.S.\ Army was known to the Axis forces for having
the most efficient artillery to date, acknowledging that they were outplayed each time not by building bigger guns, but by
focusing on the art of perfecting the current weaponry. If one were to visit the Field Artillery museum at Fort Sill, the
home of FA in the U.S military, the evolution of these systems is evident primarily by observing the transition between
primitive weaponry, like the cannons used by Napoleon Bonaparte's hordes, to the guns used in WWII, which highlights the
usefulness of have a machine that focuses on efficient recoil systems which ultimately allowed for more efficient rapid fires.
Another obvious example of the American's implementation of superior firepower in this area would be the transition mid-war
from the French 75mm and 105mm purchased after the first Great War to the famous M2/M3 105mm howitzer and the M1 75mm pack howitzer
that effectively kickstarted a renaissance of artillery implementation in Airborne, Mountain, and Jungle divisions.

\vspace{1.0\baselineskip}

The M1 75mm pack howitzer, later redesignated to the M116 75mm howitzer, could be towed by any small vehicle and allowed for
highly mobile units such as the Airborne divisions to leave combat as fast as they could enter. Due to it's modularity, the M1
could very easily be broken down and attached to parachutes and dropped behind enemy lines with infantry divisions, leading to
devastating mortality rates and countless successful battles and an almost iminent deterioration of the enemy's morale. 

\vspace{1.1\baselineskip}

The U.S \ Army was not the only major player in the field artillery game during WWII. While we were far superior, the Wermacht
devasted battlefields with artillery that seconded only ours. An interesting endeavor overseen by the German military was the
implementation of the railway gun known as ``Schwerer Gustav'', which was an 80cm gun designed specifically for sieging forts
on the French-Maginot line, which were the most heavily fortified strongholds in existence at the time. Unfortunately for the
Wermacht, the Gustav piece was not ready for deployment in France, but did get deployed against the front against the Soviet
army. The gun's first live fire in combat was during the Siege of Sevastopol where it engaged several different fortifications
on the way to the city of Sevastopol. Only using 48 shells total between the first engagements of the Soviet's outer defenses and
the city's eventual fall, the Gustav proved to be devastating against the enemy during these engagements due to it's sheer size and
size of shell, which are the biggest shells ever used in an artillery weapon. If the gun were able to maneuvered easier, the war
could have turned the tide of the war toward the Axis, as it and other weapons developed like it could have been deployed rapidly
against the Allied forces. The biggest downfall of the weapon was the time it took to deploy it, and manpower needed to operate it.
For example, for the Siege of Sevastopol, it took 4,000 men and five weeks to deploy it to it's firing position as well as 500 men
required to operate it once it was in place.

\vspace{1.0\baselineskip}

Based on the results of the Second Great War and the rapid development of artillery for all sides during it, the Allied power's
development of existing artillery in order to make them more efficient and more maneuverable proved to outplay the Axis' interest
in ``the bigger the better'' philosophy. The biggest lesson learned by the Allies post mortem was that the need for artillery that
could be rapidly deployed with infantry and airborne units was necessary to win battles against enemies not just in Europe, but
anywhere in the world.

\end{document}
